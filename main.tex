\documentclass{llncs}


%\usepackage{amsmath,amssymb}

%\usepackage{graphicx}
\usepackage{pgfplots}
%\usepackage{graphics}
%\usepackage{placeins}
%\usepackage{kpfonts}

\newcommand{\ignore}[1]{}

\newcommand{\dil}[1]{\notethis{green}{Dilian}{#1}}
\newcommand{\kla}[1]{\notethis{blue}{Klaus}{#1}}
\newcommand{\mar}[1]{\notethis{red}{Marieke}{#1}}
\newcommand{\ros}[1]{\notethis{orange}{Rosemary}{#1}}

\newcommand{\notethis}[3]{
  \color{#1}
  \vspace{0.3cm}
  \noindent\makebox[\linewidth]{\rule{\paperwidth}{0.4pt}}\\
  {\sc #2 - } #3\\
  \noindent\makebox[\linewidth]{\rule{\paperwidth}{0.4pt}}\\
  \vspace{0.3cm}
  \color{black}
}

\newcommand{\red}[1]{{\color{red} #1}}
\newcommand{\green}[1]{{\color{green} #1}}
\newcommand{\blue}[1]{{\color{blue} #1}}
\newcommand{\orange}[1]{{\color{orange} #1}}


\title{
  Static and Runtime Verification,\\ 
  Competitors or Friends?\\
  (Track Summary)
}

\author{
Dilian Gurov\inst{1}
\and
Klaus Havelund\inst{2}\thanks{The research performed by this author was carried out at Jet Propulsion Laboratory, California Institute of Technology, under a contract with the National Aeronautics and Space Administration.}
\and 
Marieke Huisman\inst{3}
\and
Rosemary Monahan\inst{4}
}
\institute{
KTH Royal Institute of Technology, Sweden,\\ \email{dilian@kth.se} 
\and
Jet Propulsion Laboratory, USA\\ \email{klaus.havelund@jpl.nasa.gov}
\and
University of Twente, The Netherlands \\ \email{m.huisman@utwente.nl}
\and
Maynooth University, Ireland\\ \email{Rosemary.Monahan@nuim.ie}
}

\begin{document}

\maketitle

\section{Motivation and Goals}

Over the last years, significant progress has been made both on dynamic and static techniques to increase the quality of software. Within this track, we would like to investigate if and how we can leverage these techniques by combining them. Questions that will be addressed are for example: what can runtime verification bring to make static verification more cost effective, and what can static verification bring to make runtime verification more precise? The session will consist of several presentations on experiences combining the two techniques, and panel discussions.


\section{Contributions}


\section{Prospects}

\kla{Note sure we need this section.}

\bibliography{bib}
\bibliographystyle{abbrv}

\end{document}
