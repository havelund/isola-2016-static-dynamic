
% -----------------------------------
% paper # 47
% -----------------------------------

Bodden, Pun, Steffen, Stolz, and Wickert
\cite{isola-2016-bodden}
({\em Information Flow Analysis for Go})
present parts of the theory and implementation of 
an information flow analysis of Go programs. The purpose is
to detect the flow of so-called tainted values, from untrusted sources (such as reading from input) to so-called sinks, 
which represent locations where such untrusted data should not end up.  Go allows for concurrent programming via channels, requiring
special techniques. It is discussed how dynamic analysis can be
applied, in addition to the  static analysis, to monitor execution paths, that due to the conservative static analysis cannot be determined safe. An option is to stop the execution of
the program when a tainted datum is about to reach a sink.
A dynamic coverage tool can also provide information
as to how many of these potentially unsafe paths have been executed and verified.
