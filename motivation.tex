
\section{Motivation and Goals}

Over the last years, significant progress has been made both on dynamic and static techniques to increase the quality of software. Within this track, we would like to investigate if and how we can leverage these techniques by combining them. Questions that will be addressed are for example: what can runtime verification bring to make static verification more cost effective, and what can static verification bring to make runtime verification more precise? 

The session will consist of several presentations on experiences combining the two techniques, and panel discussions. When preparing this session, we have aimed at finding a balance between static and runtime verification backgrounds of the presenters. This is also reflected by the papers associated to this track. There are several papers that first to verify as much as possible by static verification, and then use run-time verification for the properties that cannot be verified statically. There is another group of papers that uses static program information to generate appropriate run-time checks. Finally, a last group of papers discusses program specification techniques from static verification, and how they can be made suitable for runtime verification, or the other way round.

During the conference, three panel discussions on this topic are planned. The first panel focus on how run-time verification can benefit from static verification; the second panel will focus on the opposite question: how can static verification benefit from run-time verification; and the last panel will discuss future research directions in this area. Concrete topics that will be discussed include the limitations and benefits of each approach, how we can combine efforts to benefit verification, what are the overheads/benefits of combining efforts, industrial application in each area, industrial needs, etc.



\kla{This section should be longer.}